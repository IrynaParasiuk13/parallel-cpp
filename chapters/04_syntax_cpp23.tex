\chapter{The Syntax in C++23}

\section{Threads, Tasks, Coroutines}

\subsection{C++11 thread}

C++11 had a very basic support for threads, in terms of \cpp{std::thread} of header \cpp{thread}. The thread starts running once the constructor is called. The object is not CopyConstructible nor CopyAssignable. 
The challenge in C++11 threads is that one needs to call \cpp{join} or \cpp{detach} (joining threads ensures synchronization and waits for completion, while detaching threads allows them to run independently without synchronization or waiting) prior to the destructor being called. If neither was called, the program was \cpp{std::abort}ed. 
Prior to calling either, one needs to check whether the thread is \cpp{joinable()}.
At the same time, it was almost impossible to handle exceptions while being able to call \cpp{join} correctly. 
%std::promise
The use of the C++11 \cpp{thread} is thus considered harmful and we will present only two short examples.

\raggedbottom
\begin{codebox}[]{\href{https://godbolt.org/z/nqdn4YvEM}{\ExternalLink}}
\footnotesize An example of the use of a C++11 thread.
\tcblower
\cppfile{code_examples/cpp23/thread1.cpp}
\end{codebox}

\raggedbottom
\begin{codebox}[]{\href{https://godbolt.org/z/1Gbf4Wo75}{\ExternalLink}}
\footnotesize An example of the use of a C++11 thread.
\tcblower
\cppfile{code_examples/cpp23/thread2.cpp}
\end{codebox}

\subsection{C++20 jthread}

C++20 adds a new class \cpp{jthread} (``joining threads''), which does not require a call to \cpp{join} or \cpp{detach}. Instead, the destructor waits for completion of the code (``joins'') automatically. 

This is an example of the ``resource acquisition is initialization'' (RAII) idiom, which is generally one of the best practices in C++. In RAII, the resource allocation is tied to an object's lifetime and is hence a class invariant. When a resource is acquired, it is allocated in the constructor, and when the object is destroyed, the resource is automatically released in the destructor. There is no risk of a resource leak. 

\raggedbottom
\begin{codebox}[]{\href{https://godbolt.org/z/eWrfrsd7K}{\ExternalLink}}
\footnotesize An example of the use of jthread.
\tcblower
\cppfile{code_examples/cpp23/jthread1.cpp}
\end{codebox}

Notice that the above example would likely result in abnormal program termination if we changed jthread to std::thread. The reason for this is that with thread, we would still need to manually call `join` or `detach` to ensure proper thread management. If we forget to do so before the `std::thread` object goes out of scope, the program could terminate abnormally.

Additionally, when using standard output, it is recommended to wrap it in a `syncstream`. A `syncstream` ensures thread-safe access to the standard output, preventing data races and ensuring correct output when multiple threads are involved.

\raggedbottom
\begin{codebox}[]{\href{https://godbolt.org/z/rr1GrWqfP}{\ExternalLink}}
\footnotesize An example of the use of jthread.
\tcblower
\cppfile{code_examples/cpp23/jthread2.cpp}
\end{codebox}

Rather commonly, one uses the lambda function to define the thread. (This is the \cpp{[]()}.) 

\raggedbottom
\begin{codebox}[]{\href{https://godbolt.org/z/7M7h5nEeM}{\ExternalLink}}
\footnotesize An example of the use of jthread.
\tcblower
\cppfile{code_examples/cpp23/jthread3.cpp}
\end{codebox}

Passing arguments to threads is, nevertheless, very useful. Notably, when we pass the first argument of type \cpp{std::stop_token token}, we request the thread to stop its execution by calling \cpp{request_stop()} on the jthread object:

\raggedbottom
\begin{codebox}[]{\href{https://godbolt.org/z/rMGGPP88q}{\ExternalLink}}
\footnotesize An example of the use of jthread.
\tcblower
\cppfile{code_examples/cpp23/jthread4.cpp}
\end{codebox}

%/ This section is not really clear, honestly, but I don't know, how to fix it
More complicated procedures for the stopping of the thread possible. One can define \cpp{std::stop_callback} object inside the thread, whose constructor takes the stop token (\cpp{std::stop_token}) and a function. The function (in the example below another lambda) gets executed, when the thread is requested to stop via the \cpp{std::stop_token}:

\raggedbottom
\begin{codebox}[]{\href{https://godbolt.org/z/hE4rMznWq}{\ExternalLink}}
\footnotesize An example of the use of jthread.
\tcblower
\cppfile{code_examples/cpp23/jthread5.cpp}
\end{codebox}

(For a substantially more complex example, see \url{https://en.cppreference.com/w/cpp/thread/stop_callback}.)

% The following needs some more work, yet.
%\raggedbottom
%\begin{codebox}[]{\href{https://godbolt.org/z/cGeMrdann}{\ExternalLink}}
%\footnotesize An example of the use of jthread.
%\tcblower
%\cppfile{code_examples/cpp23/jthread6.cpp}
%\end{codebox}

%\subsection{Tasks}

\subsection{Coroutines}

In C++23 and subsequent versions, we can define coroutines, which provide a way to pause the execution of a function, return control to the calling code, and later resume from where it left off, without blocking the entire program. To get a preview, consider class template std::generator in header generator presents a view of the elements yielded by the evaluation of a coroutine:

\raggedbottom
\begin{codebox}[]{}
\footnotesize An example of the use of coroutines, which currently does not compile in trunk of gcc or clang or MSVC 19.
\tcblower
\cppfile{code_examples/cpp23/coroutines1.cpp}
\end{codebox}

Already, there are three new keywords:
\begin{itemize}
\item \cpp{co_await awaiter} suspends computation and block the co-routine until the computation is resumed by another co-routine calling ``resume'' method of the present coroutine. In the process, it tests whether it is possible to suspend the computation using an awaiter (such as \cpp{std::suspend_always{};}) and, if so, saves all local variables to a heap-allocated handle.
\item \cpp{co_yield} yields a value and suspends computation as above, and  
\item \cpp{co_return} returns a value. (There is no notion of an optional return type in-built.)
\begin{end}

Unfortunately, defining the coroutine in C++20 take some more effort. In particular, it requires:
\begin{itemize}
\item defining the behaviour of the coroutine, which is known as a \cpp{promise} (different from \cpp{std::promise}), and requires one returns the type used to access the state of the coroutine on the heap, which is known as the handle,  
\item defining how to store the state of the coroutine on the heap, using template class \cpp{std::coroutine_handle} parametrized by the promise
\end{itemize}
Clearly, one needs to declare one, define the other, and then return to declare the first one. We will see how to do this later.
Optionally, we can also define an awaiter, which controls suspension and resumption behaviour.

Another difficulty in using coroutines is the fact that the coroutine may live longer than the scope it has been called from. It is hence \emph{not} advisable to pass by reference, except perhaps \cpp{std::ref} or \cpp{std::cref}. One can either pass by value or pass, e.g., \cpp{std::unique_ptr}:

\raggedbottom
\begin{codebox}[]{}{}
\footnotesize An example of the use of coroutines, which currently does not compile in GCC 12.2.
\tcblower
\cppfile{code_examples/cpp23/coroutines2.cpp}
\end{codebox}

Below, we will see two examples of the use of coroutines that compile with GCC 11 and 12 (enable -std=c++2b -fcoroutines).
These focus on defining the promise class to be used with template class \cpp{std::coroutine_handle<promise>}.
The promise class defines the behaviour of the coroutine by implementing methods:
\begin{itemize}
\item \cpp{coroutine get_return_object()} is called to initialize the coroutine and create the coroutine handle, which can be the rather formulaic \cpp{coroutine_handle::from_promise(*this)};} 
\item \cpp{std::suspend_always initial_suspend()}, suggests whether the coroutine starts right after initialization (\cpp{std::suspend_never()} ) or upon resumption (\cpp{std::suspend_always()}). (Both awaiterers are described below.)
\item \cpp{std::suspend_always final_suspend() noexcept}, which can be rather formulaic \cpp{std::suspend_always()}
\item \cpp{void return_void()} or \cpp{void return_value(const auto& value)}, which is called upon reaching the end of the coroutine and upon reaching \cpp{co_return}. The latter (\cpp{return_value}) often just stores the result locally. 
\item \cpp{void unhandled_exception()}, which can be rather formulaic \cpp{std::terminate()}, or can save the exception via \cpp{std::current_exception()}.
\end{itemize}

The promise class is instantiated for each instance of the coroutine, and its methods are called as follows:

\raggedbottom
\begin{codebox}[]{}
\footnotesize A schema of the coroutine, in terms of its calls of the methods of the promise class.
\tcblower
\cppfile{code_examples/cpp23/coroutines_schema.h}
\end{codebox}

Once we have a promise class, we can specialize template class \cpp{std::coroutine_handle}, which can be seen as the equivalent of a pointer and its method ``destroy'' as equivalent to a ``free'', and use the handle specialized to our own promise class to define a promise class:

\raggedbottom
\begin{codebox}[]{\href{https://godbolt.org/z/hP37s5jjh}{\ExternalLink}}
\footnotesize An example of the use of coroutines.
\tcblower
\cppfile{code_examples/cpp23/coroutines3.cpp}
\end{codebox}

Sometimes, we also store the promise type in a \cpp{promise_type} type member, and disable (\cpp{= delete}) copy and move constructors. 

\subsubsection{Awaiters}

Finally, let us consider awaiters, which can be called when a coroutine is suspended or resumed. It provides the necessary machinery to wait for an asynchronous operation to complete and resume the coroutine when the result is available.
Key methods of an awaiter include:
\begin{itemize}
\item \cpp{await_ready()} is called immediately before suspension of a coroutine. If it returns \cpp{true}, the coroutine will not be suspended. 
\item \cpp{await_suspend(handler)} is called immediately after the suspension of the coroutine. The \cpp{handler} of type \cpp{std::coroutine_handle} can be used to pass the state of the coroutine (e.g., to another thread). 
\item \cpp{await_resume()} is called when the coroutine is resumed after a successful suspension. If it returns a value, this will be returned by the \cpp{co_await} routine. 
\end{itemize}
The two awaiters we have seen so far (\cpp{std::suspend_never()} and \cpp{std::suspend_always()}) just returned boolean constants in \cpp{await_ready()}:

\raggedbottom
\begin{codebox}[]{\href{https://godbolt.org/z/sK884T7hh}{\ExternalLink}}
\footnotesize An example of two standard awaiters.
\tcblower
\cppfile{code_examples/cpp23/awaiters1.h}
\end{codebox}

By defining \cpp{await_transform()} in the promise type, the compiler will use \cpp{co_await promise.await_transform(<expr>)} instead of any call of \cpp{co_await <expr>} in the coroutine. 

\raggedbottom
\begin{codebox}[]{\href{https://godbolt.org/z/b5a94YeKx}{\ExternalLink}}
\footnotesize An example defining a Generator (1/2).
\tcblower
\cppfile{code_examples/cpp23/coroutines4.h}
\end{codebox}

\raggedbottom
\begin{codebox}[]{\href{https://godbolt.org/z/b5a94YeKx}{\ExternalLink}}
\footnotesize An example defining a Generator (2/2).
\tcblower
\cppfile{code_examples/cpp23/coroutines4.cpp}
\end{codebox}

Eventually, in C++23 and C++26, the support for message-passing architectures based on \url{https://github.com/lewissbaker/cppcoro} should be available. For a small sample, see a custom implementation here:

\raggedbottom
\begin{codebox}[]{\href{https://godbolt.org/z/KhYaGPn7e}{\ExternalLink}}
\footnotesize An example of the use of coroutines in message-passing architectures (1/3).
\tcblower
\cppfile{code_examples/cpp23/coroutines5a.h}
\end{codebox}

\raggedbottom
\begin{codebox}[]{\href{https://godbolt.org/z/5vEsW6dKE}{\ExternalLink}}
\footnotesize An example of the use of coroutines in message-passing architectures (2/3)
\tcblower
\cppfile{code_examples/cpp23/coroutines5b.h}
\end{codebox}

\raggedbottom
\begin{codebox}[]{\href{https://godbolt.org/z/vEaM8PM9s}{\ExternalLink}}
\footnotesize An example of the use of coroutines in message-passing architectures (3/3)
\tcblower
\cppfile{code_examples/cpp23/coroutines5.cpp}
\end{codebox}

For further nice examples, see also Boost Asio, e.g.,
\url{https://www.boost.org/doc/libs/1_78_0/doc/html/boost_asio/example/cpp20/channels/throttling_proxy.cpp}, 
as discussed, e.g., at
\url{https://www.youtube.com/watch?app=desktop&v=ZNttI_WswMU&ab_channel=ACCUConference}.
For more details, see Simon Toth's Complete guide at \url{https://itnext.io/c-20-coroutines-complete-guide-7c3fc08db89d}.
For the technical specification, see \url{https://github.com/GorNishanov/coroutines-ts}.

\section{Synchronisation Primitives}

\subsection{Atomic Variables}
\label{sec:atomiccpp23}

Since C++11, there is an excellent support for atomic variables (hey ensure that operations performed on the variable are executed atomically, meaning they are indivisible and cannot be interrupted by other threads) in header \cpp{<atomic>}. 
The primary template can be instantiated with types that are TriviallyCopyable, CopyConstructible, and CopyAssignable.
First, let us consider two simple examples:

\raggedbottom
\begin{codebox}[]{\href{https://godbolt.org/z/4Gjjrbjso}{\ExternalLink}}
\footnotesize An example of the use of atomic variables.
\tcblower
\cppfile{code_examples/cpp23/atomic1.cpp}
\end{codebox}

\raggedbottom
\begin{codebox}[]{\href{https://godbolt.org/z/K4q5n3xh5}{\ExternalLink}}
\footnotesize An example of the use of atomic variables.
\tcblower
\cppfile{code_examples/cpp23/atomic2.cpp}
\end{codebox}

Finally, let us consider two, more complete examples of a stack:

\raggedbottom
\begin{codebox}[]{\href{https://godbolt.org/z/vs5bKzjxe}{\ExternalLink}}
\footnotesize A stack implemented using atomic variables.
\tcblower
\cppfile{code_examples/cpp23/atomic5.cpp}
\end{codebox}

We will elaborate upon this later in Chapter \ref{ch:datastructures}. 

\subsection{Mutexes and Locks}

Standard Template Library in header \cpp{<mutex>}	provides multiple mutexes (of type \cpp{BasicLockable} that implement \cpp{lock} and \cpp{unlock} methods): \cpp{mutex}, \cpp{recursive_mutex}, \cpp{timed_mutex}, \cpp{recursive_timed_mutex}, 
and \cpp{unique_lock}. 

A good practice for the use of mutexes is to lock them via the RAII idiom. Since C++11, this is available as \cpp{std::unique_lock} and \cpp{std::lock_guard}, and since C++17 \cpp{scoped_lock} in header \cpp{<mutex>}.
Crucially, using \cpp{scoped_lock} provides the ability to lock multiple mutexes at once, avoiding deadlock.
One may hence advise to use one or more \cpp{mutex} with a \cpp{scoped_lock} on top. 

\raggedbottom
\begin{codebox}[]{\href{https://godbolt.org/z/zeM6xMz84}{\ExternalLink}}
\footnotesize An example of the use of a unique lock.
\tcblower
\cppfile{code_examples/cpp23/mutex1.cpp}
\end{codebox}

%\raggedbottom
%\begin{codebox}[]{\href{https://godbolt.org/z/Tvxj9Yczj}{\ExternalLink}}
%\footnotesize Locking multiple mutexes at once.
%\tcblower
%\cppfile{code_examples/cpp23/lock_guard.h}
%\end{codebox}

A slightly more verbose example considers:

\raggedbottom
\begin{codebox}[]{\href{https://godbolt.org/z/qeYzaz6We}{\ExternalLink}}
\footnotesize An example of the use of a unique lock.
\tcblower
\cppfile{code_examples/cpp23/mutex2.cpp}
\end{codebox}

\subsection{Barrier}

Since C++20, there is support for barriers in header \cpp{<barrier>}. They allow threads to synchronize and coordinate their progress by providing a designated point in the code where all participating threads must reach before any of them can proceed further. When a thread reaches a barrier, it will be blocked until all other threads in the same synchronization group also reach the barrier. Once all threads have arrived, the barrier is released, and all threads are allowed to continue executing. The constructor takes an integer value, which is the number of threads that the barrier is expected to block. The key methods are \cpp{arrive_and_wait()} and \cpp{arrive_and_drop()}. The former functions as one would expect. The latter decrements the initial expected count for all uses by one, as if one thread could never reach the barrier subsequently. This can be very useful in error management:

\raggedbottom
\begin{codebox}[]{\href{https://godbolt.org/z/fT17E5P4x}{\ExternalLink}}
\footnotesize An example of the use of a barrier.
\tcblower
\cppfile{code_examples/cpp23/barrier1.cpp}
\end{codebox}

More complicated uses of barriers may use the template parameter \cpp{CompletionFunction} and have a callable executed whenever the barrier is hit (reaches zero):

\raggedbottom
\begin{codebox}[]{\href{https://godbolt.org/z/McM63hnn4}{\ExternalLink}}
\footnotesize An example of the use of a barrier.
\tcblower
\cppfile{code_examples/cpp23/barrier2.cpp}
\end{codebox}

\section{Algorithms in the Standard Template Library}

Since C++17, there is an excellent Parallel Standard Template Library in header \cpp{<algorithm>}.

\subsection{For Each}
\label{sec:foreachcpp23}

The most useful algorithm from the Standard Template Library (STL) in terms of parallel programming is surely for each:

\raggedbottom
\begin{codebox}[]{\href{https://godbolt.org/z/vrbcYdM3o}{\ExternalLink}}
\footnotesize An example of the use of for each.
\tcblower
\cppfile{code_examples/cpp23/for_each_code_parallel.h}
\end{codebox}

As in the serial version of STL, the callable within for each is permitted to change the state of elements, if the underlying range is mutable, but cannot invalidate iterators.

\subsection{Reduce}

Similarly useful is the reduce operation (also known as fold, accumulate, aggregate, compress, or inject). Indeed, a whole approach to parallel programming (Map Reduce) is named after the algorithm. There, one applies an associative operation to each piece of data to obtain a partial result, and then (perhaps using a binary-tree reduction, which also may enforce an order which may or may not be desirable) obtains the final result by applying the same associative operation to the partial results. The binary-tree reduction makes it possible to utilize $O(\log(n))$ rounds of computation on $n$ processors. 

\raggedbottom
\begin{codebox}[]{\href{https://godbolt.org/z/bG5q3GxzK}{\ExternalLink}}
\footnotesize An example of the use of reduce.
\tcblower
\cppfile{code_examples/cpp23/reduce_code.h}
\end{codebox}

\subsection{Merge}

Finally, in implementing parallel sorting algorithms, we will utilize the parallel merge operation:

\raggedbottom
\begin{codebox}[]{\href{https://godbolt.org/z/477sfja8z}{\ExternalLink}}
\footnotesize An example of the use of a merge.
\tcblower
\cppfile{code_examples/cpp23/merge_par_code.h}
\end{codebox}
